\begin{titlepage}
%\large 
    \begin{center}
	Министерство науки и высшего образования Российской Федерации\\Санкт-Петербургский Политехнический Университет Петра Великого \\ Физико-механический институт \\ \textbf{Высшая школа фундаментальных физических исследований} 
\end{center}
\begin{flushright}
\begin{tabular}{l}
Работа допущена к защите\\Директор ВШФФИ \vspace{0.3cm} \\
\underline{\hspace{3.5cm}} В. В. Дубов \vspace{0.5cm}\\
«\underline{\hspace{1cm}}»\underline{\hspace{3cm}} 2024 г.
\end{tabular}
\end{flushright}
\vspace{0.3cm}
{ \begin{center}
\vspace{0.3cm}
\MakeUppercase{\textbf{выпускная квалификационная работа \\ магистерская диссертация}}\\
\vspace{0.3cm}

\MakeUppercase{\textbf{Асимметрия Сиверса в полуинклюзивном глубоко неупругом рассеянии лептонов на поперечно поляризованном протоне}}
\\
\vspace{0.5cm}
по направлению подготовки 03.04.02 «Физика»
\\
по образовательной программе
\\
$03.04.02\_03$ «Физика ядра и элементарных частиц в фундаментальных и медицинских исследованиях»
\end{center}
}
\vspace{0.3cm}

\begin{flushleft}
Выполнил \\
студент гр. 5040302/20301 \hspace{0.3cm} %\hspace{1.9 cm} 
\hfill  Е.В. Музяев
\vspace{0.5cm}


Научный руководитель\\
профессор ВШФФИ, д.ф.-м.н.
\hspace{0.1 cm} \hspace{1.8 cm} 
\hfill  {Я.А. Бердников}
\vspace{0.5cm}

Консультант \\
по нормконтролю \hspace{2.1 cm} \hspace{2 cm} \hfill  {И.Г. Голиков}
\end{flushleft}


\begin{center}
	Санкт-Петербург
\end{center}
\begin{center}
    2024
\end{center}
\newpage
\pagestyle{empty}
\section*{РЕФЕРАТ}
\begin{center}
    На n страниц, k рисунков, l таблиц.
\end{center}
КЛЮЧЕВЫЕ СЛОВА: Глубоко неупругое рассеяние
\\


\section*{ABSTRACT}
\begin{center}
    n pages, k pictures, l table
\end{center} 
KEY WORDS: DEEP INELASTIC SCATTERING
\\


\end{titlepage}	