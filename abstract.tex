\newpage


\thispagestyle{empty}
\begin{center}
	\textbf{\MakeUppercase{Реферат}}
\end{center}
%\section*{РЕФЕРАТ}
\begin{center}
	На n страниц, 8 рисунков, 0 таблиц.
\end{center}

КЛЮЧЕВЫЕ СЛОВА: Глубоко неупругое рассеяние, поляризация, асимметрия Сиверса, адронизация, партонная модель, PYTHIA8.

Тема выпускной квалификационной работы: "Асимметрия Сиверса в полуинклюзивном глубоко неупругом рассеянии лептонов на поперечно поляризованном протоне".

Данная работа посвящена исследованию и разработке алгоритма расчёта асимметрии Сиверса адронов в полуинклюзивном глубоко неупругом рассеянии лептонов на поперечно поляризованном протоне. Был проделан обзор литературных источников по данной теме. На основе теоретических моделей был создан алгоритм расчёта асимметрии Сиверса с использованием Монте-Карло генератора событий PYTHIA8 и плагина Stringspinner. На основе алгоритма расчёта были получены значения асимметрии Сиверса в кинематике эксперимента COMPASS. Было проведено сравнение результатов с экспериментальными данными, которое показало, что полученные с помощью расчёта данные совпадают с точностью до погрешности с экспериментальными. Исходя из этого, можно сделать вывод, что данный метод можно использовать для расчёта асимметрии Сиверса.
\thispagestyle{empty}


\begin{center}
	\textbf{\MakeUppercase{Abstract}}
\end{center}
\begin{center}
	n pages, 8 pictures, 0 table
\end{center} 

KEY WORDS: Deep inelastic scattering, polarization, Sivers asymmetry, hadronization, parton model, PYTHIA8.

The subject of the graduate qualification work is "Sivers Asymmetry in semi-inclusive deep inelastic scattering of leptons on transversely polarized proton". 

The given work is devoted to research and creating calculation algorithm of Sivers asymmetry in deep inelastic scattering between lepton and transversely polarized proton. A literature study has been conducted. A calculation algorithm of Sivers asymmetry based on Monte-Carlo event generator PYTHIA8 with Stringspinner plug-in has been created. Calculations of Sivers asymmetry in COMPASS kinematics has been done. Results of calculation algorithm has been compared to experimental data. Comparison has been shown coincidence between COMPASS data and modeling data up to the limits of accuracy of the measurements.  
 
\newpage

\chapter*{ОБОЗНАЧЕНИЯ И СОКРАЩЕНИЯ}
\thispagestyle{empty}
В настоящей работе применяют следующие обозначения и сокращения: \\
ГНР -- глубоко неупругое рассеяние; \\
ПГНР -- полуинклюзивное глубоко неупругое рассеяние; \\
ФФ -- функции фрагментации; \\
PDF -- партонные функции распределения; \\ 
TMD -- партонные функции распределения, зависящие от поперечного импульса; \\
EMC -- European Muon Collaboration; \\
COMPASS -- COmmon Muon and Proton Apparatus for Structure and
Spectroscopy; \\
HERA -- Hadron Electron Ring Accelerator; \\
HERMES -- HERA MEasurement of Spin; \\
SLAC -- Stanford Linear Accelerator Center.
